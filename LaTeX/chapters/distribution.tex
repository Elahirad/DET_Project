\فصل{توزیع‌های آشکارساز}
در این قسمت، ابتدا خاصیت \لر{CFAR}\پانویس{Constant false alarm rate} آشکارساز معرفی شده را بررسی می‌کنیم. سپس، توزیع مجانبی 
$T_R$
را تحت فرض‌های 
$H_0$
و
$H_1$
به دست می‌آوریم. از آن‌جایی که
$T_R$
محدود به بازه‌ی
$[0, nm(m-1)]$
است، می‌توانیم یک توزیع بتا را برای توزیع تقریبی آشکارساز انتخاب کنیم. نحوه به دست آوردن توزیع به این صورت است که ابتدا، ممان های مرتبه اول و دوم آشکارساز را به دست می‌آوریم و با ممان های متناظر توزیع بتا مطابقت می‌دهیم تا پارامترها را پیدا کنیم.
\قسمت{خاصیت \لر{CFAR}}

برای بررسی ویژگی CFAR آشکارساز پیشنهادی، از نظریه‌ی \لر{invariant} استفاده می‌کنیم.  
فرض کنید 
$\mathbf{\Sigma}'$
یک ماتریس قطری با درایه‌های قطری نامعلوم و مثبت باشد. برای اثبات ویژگی CFAR کافی است دو نکته را تحت $H_{0}$ نشان دهیم:

\begin{enumerate}
	\item کوانتیزه‌سازی یک‌بیتی تبدیل
	$\mathbf{\Sigma}'^{1/2}\mathbf{x}(t)$،
	که با 
	$\mathcal{Q}(\mathbf{\Sigma}'^{1/2}\mathbf{x}(t))$ 
	نشان داده می‌شود، متعلق به همان خانواده‌ی توزیع داده‌های اولیه‌ی یک‌بیتی
	$\mathbf{y}(t)$ 
	است.
	\item آشکارساز پیشنهادی
	 $\mathcal{Q}(\mathbf{\Sigma}'^{1/2}\mathbf{x}(t))$ 
	 را دقیقاً به همان نتیجه‌ای نگاشت می‌کند که
	 $\mathbf{y}(t)$
	 را می‌کند.
\end{enumerate}

از آن‌جا که 
$\mathbf{\Sigma}'$ 
قطری با درایه‌های مثبت است و با توجه به خاصیت
 $\mathrm{sign}(ax)=\mathrm{sign}(x)$
 برای
  $a>0$،
  داریم
\begin{align}
	\mathcal{Q}(\mathbf{\Sigma}'^{1/2}\mathbf{x}(t))
	&=\mathrm{sign}\!\big(\mathbf{\Sigma}'^{1/2}\mathrm{Re}(\mathbf{x}(t))\big)
	+ j\,\mathrm{sign}\!\big(\mathbf{\Sigma}'^{1/2}\mathrm{Im}(\mathbf{x}(t))\big) \notag \\
	&=\mathrm{sign}\!\big(\mathrm{Re}(\mathbf{x}(t))\big)
	+ j\,\mathrm{sign}\!\big(\mathrm{Im}(\mathbf{x}(t))\big) \notag \\
	&=\mathbf{y}(t). \label{eq:cfar1}
\end{align}
بنابراین 
$\mathcal{Q}(\mathbf{\Sigma}'^{1/2}\mathbf{x}(t))$
و 
$\mathbf{y}(t)$
دارای توزیع یکسانی هستند.  

اکنون مقدار آماره‌ی آزمون رائو برای داده‌های تبدیل‌شده بررسی می‌شود:
\begin{align}
	T_{R}\!\left(\mathcal{Q}(\mathbf{\Sigma}'^{1/2}\mathbf{X})\right)
	&=\frac{n}{2}\sum_{i,j=1,\; i<j}^{m}
	\left|\frac{1}{n}\sum_{t=1}^{n}
	\mathbf{Q}(\sigma'_{i}x_{i}(t))\,\mathbf{Q}(\sigma'_{j}x_{j}(t))^{*}\right|^{2} \notag \\
	&=\frac{n}{2}\sum_{i,j=1,\; i<j}^{m}
	\left|\frac{1}{n}\sum_{t=1}^{n}y_{i}(t)\,y_{j}^{*}(t)\right|^{2} \notag \\
	&=T_{R}(\mathbf{Y}), \label{eq:cfar2}
\end{align}
که در آن
$\mathbf{Y}=[\mathbf{y}(1),\ldots,\mathbf{y}(n)]$
و 
$\sigma'_{i}$
درایه‌ی
$(i,i)$ 
ماتریس
$\mathbf{\Sigma}'$ 
است.  

به طور مشابه، برای آزمون EMR یک‌بیتی داریم:
\begin{align}
	T_{O}
	&=1+\frac{1}{m}\sum_{i,j=1,\; i<j}^{2m}
	|\hat r_{\tilde{\mathbf{y}}}(i,j)|^{2}, \label{eq:cfar3}
\end{align}
که در آن 
$\hat r_{\tilde{\mathbf{y}}}(i,j)$ 
درایه‌ی $(i,j)$ ماتریس کوواریانس یک‌بیتی گسترش‌یافته
$\hat{\mathbf{R}}_{\tilde{\mathbf{y}}}$ 
است.  
اگر بردار 
$\tilde{\mathbf{y}}_{\mathrm{tra}}(t)$ 
را به صورت زیر تعریف کنیم
\[
\tilde{\mathbf{y}}_{\mathrm{tra}}(t)=
\big[\,\mathrm{Re}(\mathcal{Q}(\mathbf{\Sigma}'^{1/2}\mathbf{x}(t)))^{T},~
\mathrm{Im}(\mathcal{Q}(\mathbf{\Sigma}'^{1/2}\mathbf{x}(t)))^{T}\,\big]^{T},
\]
خواهیم داشت
\begin{align}
	\hat{\mathbf{R}}_{\tilde{\mathbf{y}}_{\mathrm{tra}}}
	&=\frac{1}{n}\sum_{t=1}^{n}\tilde{\mathbf{y}}_{\mathrm{tra}}(t)\tilde{\mathbf{y}}_{\mathrm{tra}}^{T}(t)
	=\frac{1}{n}\sum_{t=1}^{n}\tilde{\mathbf{y}}(t)\tilde{\mathbf{y}}^{T}(t)
	=\hat{\mathbf{R}}_{\tilde{\mathbf{y}}}. \label{eq:cfar4}
\end{align}
در نتیجه
\[
T_{O}\!\left(\mathcal{Q}(\mathbf{\Sigma}'^{1/2}\mathbf{X})\right)=T_{O}(\mathbf{Y}).
\]

\noindent\textbf{نتیجه‌گیری:}  
بنابراین هم روش پیشنهادی و هم آزمون EMR یک‌بیتی حتی در شرایط نامعینی واریانس نویز، آستانه‌ی آشکارسازی ثابتی را حفظ می‌کنند.  
به بیان دیگر، هر دو روش دارای ویژگی CFAR هستند. این خاصیت نیز با شبیه‌سازی‌های فصل‌های بعد تأیید می‌شود.
\قسمت{توزیع تحت فرض $H_0$}
برای آن که آشکارساز به بازه‌ی 
$[0, 1]$
نگاشت شود، آماره‌ی جدید
$T'_R$
را به صورت زیر تعریف می‌کنیم.
\begin{equation}
	T'_R=\frac{1}{nm(m-1)}T_R
\end{equation}
 ممان‌های مرتبه اول و دوم این آماره، در قضیه زیر داده شده‌اند.

\begin{قضیه}
	تحت فرض
	$H_0$،
	میانگین و واریانس 
	$T'_R$
	به صورت زیر هستند :
	\begin{equation}
		\mu_0=\frac{1}{n} \label{eq:39}
	\end{equation}	
	\begin{equation}
		\sigma_0^2=\frac{2(n-1)}{m(m-1)n^3} \label{eq:40}	
	\end{equation}
\end{قضیه}
\begin{اثبات}
	
	از آن‌جا که مشاهدات در زمان‌های مختلف مستقل هستند و هر مؤلفه‌ی $\tilde y_{a}(t)$ تنها می‌تواند مقادیر $\pm 1$ بگیرد، داریم
	\begin{align}
		\mathbb{E}\!\left[\prod_{t=1}^{n}\prod_{a=1}^{2m}\big(\tilde y_{a}(t)\big)^{\eta_{at}}\right]
		&=\prod_{t=1}^{n}\mathbb{E}\!\left[\prod_{a=1}^{2m}\big(\tilde y_{a}(t)\big)^{\mathrm{mod}(\eta_{at},2)}\right], \label{eq:B1}
	\end{align}
	که در آن $\eta_{at}\in\mathbb{N}$ و $\mathrm{mod}(\eta,2)$ باقیمانده‌ی تقسیم $\eta$ بر $2$ است.  
	
	تحت $H_{0}$، تابع جرم احتمال $p(\tilde{\mathbf{Y}};\theta=\theta_{0})$ برابر است با
	\[
	p(\tilde{\mathbf{Y}};\theta=\theta_{0})
	=\Big(\tfrac{1}{2}\Big)^{2mn}.
	\]
	در نتیجه مؤلفه‌های $\tilde Y$ مستقل‌اند و
	\[
	\Pr\{\tilde y_{a}(t)=1\}=\Pr\{\tilde y_{a}(t)=-1\}=\tfrac{1}{2}.
	\]
	بنابراین
	\[
	\mathbb{E}\!\left[\prod_{t=1}^{n}\prod_{a=1}^{2m}\big(\tilde y_{a}(t)\big)^{\eta_{at}}\right]
	=\begin{cases}
		1, & \text{اگر همه‌ی $\eta_{at}$ها زوج باشند},\\[4pt]
		0, & \text{در غیر این صورت}.
	\end{cases}
	\]
	
	اگر $z_{ij}(t)=y_{i}(t)y_{j}^{*}(t)$ باشد. با استفاده از رابطه‌ی بالا داریم
	\begin{align}
		\mathbb{E}\!\left[z_{ij}(t_{1})z_{ij}^{*}(t_{2})\right]&=4\,\delta_{t_{1}t_{2}}, \label{eq:B2}\\
		\mathbb{E}\!\left[z_{ij}(t_{1})z_{ij}^{*}(t_{2})z_{kl}(t_{3})z_{kl}^{*}(t_{4})\right]
		&=16\,\delta_{t_{1}t_{2}}\delta_{t_{3}t_{4}}
		+16\,\delta_{ik}\delta_{jl}\,\delta_{t_{1}t_{4}}\delta_{t_{2}t_{3}}\,(1-\delta_{t_{1}t_{2}}\delta_{t_{3}t_{4}}), \label{eq:B3}
	\end{align}
	که در آن $1\le i<j\le m,\;1\le k<l\le m,\;1\le t_{1},t_{2},t_{3},t_{4}\le n$.
	
	\medskip
	\noindent\textbf{گام ۱: محاسبه‌ی میانگین $T_{R}'$.}
	
	داشتیم‌:
	\[
	T_{R}'=\frac{1}{nm(m-1)}\,T_{R}.
	\]
	پس میانگین آن برابر است با
	\begin{align}
		\mu_{0}=\mathbb{E}[T_{R}']
		&=\frac{1}{2m(m-1)}\sum_{i,j=1,\; i<j}^{m}\mathbb{E}\!\left[|\hat r_{ij}|^{2}\right] \notag \\
		&=\frac{1}{2m(m-1)}\sum_{i,j=1,\; i<j}^{m}
		\mathbb{E}\!\left[\frac{1}{n^{2}}\sum_{t_{1},t_{2}=1}^{n}z_{ij}(t_{1})z_{ij}^{*}(t_{2})\right] \notag \\
		&=\frac{1}{2m(m-1)}\sum_{i,j=1,\; i<j}^{m}\frac{1}{n^{2}}\sum_{t_{1},t_{2}=1}^{n}\mathbb{E}[z_{ij}(t_{1})z_{ij}^{*}(t_{2})]. \label{eq:B4}
	\end{align}
	با استفاده از \eqref{eq:B2}، تنها حالت $t_{1}=t_{2}$ باقی می‌ماند و
	\[
	\mu_{0}=\frac{1}{n}.
	\]
	
	\medskip
	\noindent\textbf{گام ۲: محاسبه‌ی واریانس $T_{R}'$.}
	
	واریانس را به صورت زیر محاسبه می‌کنیم
	\begin{align}
		\sigma_{0}^{2}&=\mathbb{E}[(T_{R}')^{2}]-\mu_{0}^{2} \notag \\
		&=\frac{1}{4m^{2}(m-1)^{2}}
		\sum_{i,j,k,l=1 \atop i<j,\;k<l}^{m}\mathbb{E}\!\left[|\hat r_{ij}|^{2}\,|\hat r_{kl}|^{2}\right]-\mu_{0}^{2}. \label{eq:B5}
	\end{align}
	اکنون
	\begin{align}
		\mathbb{E}\!\left[|\hat r_{ij}|^{2}\,|\hat r_{kl}|^{2}\right]
		&=\mathbb{E}\!\left[\frac{1}{n^{4}}\sum_{t_{1},t_{2},t_{3},t_{4}=1}^{n}
		z_{ij}(t_{1})z_{ij}^{*}(t_{2})z_{kl}(t_{3})z_{kl}^{*}(t_{4})\right] \notag \\
		&=\frac{1}{n^{4}}\sum_{t_{1},t_{2},t_{3},t_{4}=1}^{n}
		\mathbb{E}\!\left[z_{ij}(t_{1})z_{ij}^{*}(t_{2})z_{kl}(t_{3})z_{kl}^{*}(t_{4})\right]. \label{eq:B6}
	\end{align}
	با جایگذاری \eqref{eq:B3} در رابطه‌ی بالا
	\begin{equation}
	\mathbb{E}\!\left[|\hat r_{ij}|^{2}\,|\hat r_{kl}|^{2}\right]
	=\mathbb{E}[|\hat r_{ij}|^{2}]\,\mathbb{E}[|\hat r_{kl}|^{2}]
	+\frac{16(n^{2}-n)}{n^{4}}\delta_{ik}\delta_{jl}. \tag{92}
	\end{equation}
	
	با جایگذاری در \eqref{eq:B5} و ساده‌سازی 
	\begin{equation}
	\sigma_{0}^{2}=\frac{2(n-1)}{m(m-1)n^{3}}. \tag{93}
	\end{equation}
	
	\medskip
	\noindent\textbf{نتیجه:}  
	تحت فرض $H_{0}$، آماره‌ی نرمالیزه‌شده‌ی $T_{R}'$ میانگین
	$
	\mu_{0}=\tfrac{1}{n}
	$
	و واریانس
	$
	\sigma_{0}^{2}=\tfrac{2(n-1)}{m(m-1)n^{3}}
	$
	دارد.
		
\end{اثبات}

تابع توزیع تجمعی (CDF) توزیع بتا به صورت زیر تعریف می‌شود:
\begin{equation}
	F(x;\alpha,\beta)=
	\frac{\Gamma(\alpha+\beta)}{\Gamma(\alpha)\Gamma(\beta)}\,
	B(x;\alpha,\beta), \tag{41}
\end{equation}
که در آن تابع بتای ناقص برابر است با
\begin{equation}
	B(x;\alpha,\beta)=\int_{0}^{x} z^{\alpha-1}(1-z)^{\beta-1}\,dz, \tag{42}
\end{equation}
و $\Gamma(x)=\int_{0}^{+\infty} t^{x-1}e^{-t}\,dt$، برای $x>0$، تابع گاما است.  

علاوه بر این، میانگین و واریانس یک توزیع بتا به صورت زیر قابل محاسبه است:
\begin{equation}
	\mu=\frac{\alpha}{\alpha+\beta}, 
	\qquad
	\sigma^{2}=\frac{\alpha\beta}{(\alpha+\beta)^{2}(\alpha+\beta+1)}. \label{eq:43}
\end{equation}

با تطبیق روابط \eqref{eq:43} با نتایج \eqref{eq:39} و \eqref{eq:40}، توزیع تقریبی \لر{Null} (فرض $H_0$) آماره‌ی $T_{R}'$ به صورت زیر به دست می‌آید:
\begin{equation}
	\Pr\{T_{R}'<\gamma\}\approx
	\frac{\Gamma(\alpha_{0}+\beta_{0})}{\Gamma(\alpha_{0})\Gamma(\beta_{0})}\,
	B(\gamma;\alpha_{0},\beta_{0}), \tag{44}
\end{equation}
که در آن
\begin{align}
	\alpha_{0}&=\frac{nm(m-1)-2}{2n}, \tag{45}\\
	\beta_{0}&=\frac{(n-1)\big[nm(m-1)-2\big]}{2n}. \tag{46}
\end{align}

\قسمت{توزیع تحت فرض $H_1$}

میانگین و واریانس 
$T'_R$
تحت فرض 
$H_1$
در قضیه‌ی زیر آمده است.
\begin{قضیه}
	تحت فرض
	$H_1$،
	میانگین و واریانس 
	$T'_R$
	به صورت زیر هستند.
	\begin{equation}
		\mu_1=\frac{1}{2m(m-1)}\sum_{i,j=1 \atop i<j}^{m}g_{ij}
	\end{equation}
	\begin{equation}
		\sigma_1^2=\frac{1}{4m^2(m-1)^2}\sum_{i,j,k,l=1 \atop i<j,k<l}^{m} \left(f_{ijkl}-g_{ijkl}\right)
	\end{equation}
	که 
	$g_{ij}$، $g_{ijkl}$
	و 
	$f_{ijkl}$
	در اثباتی که در ادامه می‌آید تعریف می‌شوند.
\end{قضیه}

\begin{اثبات}
	
	با ترکیب حل‌های بسته برای «احتمال‌های \لر{orthant} مرکزی» مرتبهٔ دوم و سوم با \linebreak
	$
	p(\tilde{\mathbf{Y}};\boldsymbol{\theta})
	= \prod_{t=1}^n p\bigl(\tilde{\mathbf{y}}(t);\boldsymbol{\theta}\bigr)
	= \prod_{t=1}^n \phi\bigl[\mathbf{S}(t)\bigr]
	$
	به نتایج زیر می‌انجامد:
	\begin{equation}
		h_{ab}\;=\;\mathbb{E}\!\big[\tilde y_{a}(t)\,\tilde y_{b}(t)\big]
		\;=\;\frac{2}{\pi}\,\arcsin \rho_{ab}, \label{eq:hab}
	\end{equation}
	و
	\begin{equation}
		\begin{aligned}
			h_{abcd}\;=\;\mathbb{E}\!\big[\tilde y_{a}(t)\tilde y_{b}(t)\tilde y_{c}(t)\tilde y_{d}(t)\big]
			\;=\;16\,P_{abcd}-1
			-\big(h_{ab}+h_{ac}+h_{ad}+h_{bc}+h_{bd}+h_{cd}\big),
		\end{aligned} \label{eq:habcd}
	\end{equation}
	که در آن \(1\le a\neq b\neq c\neq d\le 2m\) و
	\begin{equation}
		P_{abcd}\;=\;\Pr\!\left\{ \tilde x_{a}(t)>0,\;\tilde x_{b}(t)>0,\;\tilde x_{c}(t)>0,\;\tilde x_{d}(t)>0 \right\}.
	\end{equation}
	
	با توجه به تقارن‌های ساختاری
	\begin{equation}
		\rho_{i'j'}=\rho_{ij},\qquad \rho_{i'j}=-\,\rho_{ij'}, \qquad 1\le i<j\le m,
	\end{equation}
	و استفاده از \eqref{eq:B1} و \eqref{eq:hab} و \eqref{eq:habcd} داریم
	\begin{equation}
		\mathbb{E}\!\big[z_{ij}(t)\big]=\mathbb{E}\!\big[y_i(t)y_j^{*}(t)\big]
		=2\big(h_{ij}+\mathrm{i}\,h_{i'j}\big), \label{eq:Ezij}
	\end{equation}
	و امیدهای ضرب‌های دوتایی به‌صورت زیرند:
	\begin{equation}
		\mathbb{E}\!\big[z_{ij}(t)\,z_{kl}(t)\big]=
		\begin{cases}
			4\,h_{ii'jj'}, & i{=}k,\; j{=}l,\\[2pt]
			2\!\left[h_{ii'jl'}+h_{ii'j'l}+\mathrm{i}\big(h_{ii'jl}-h_{ii'j'l'}\big)\right], & i{=}k,\; j\neq l,\\[2pt]
			4\big(h_{kj}+\mathrm{i}\,h_{k'j}\big), & i{=}l,\\[2pt]
			4\big(h_{il}+\mathrm{i}\,h_{i'l}\big), & j{=}k,\\[2pt]
			2\!\left[h_{jj'ik'}+h_{jj'i'k}+\mathrm{i}\big(h_{jj'i'k'}-h_{jj'ik}\big)\right], & j{=}l,\; i\neq k,\\[2pt]
			\upsilon_{1}(i,j,k,l)-\upsilon_{2}(i,j,k,l)
			+\mathrm{i}\big[\upsilon_{3}(i,j,k,l)+\upsilon_{4}(i,j,k,l)\big], & i\neq j\neq k\neq l,
		\end{cases} \label{eq:Ezijkl}
	\end{equation}
	و
	\begin{equation}
		\mathbb{E}\!\big[z_{ij}(t)\,z^{*}_{kl}(t)\big]=
		\begin{cases}
			4, & i{=}k,\; j{=}l,\\[2pt]
			4\big(h_{lj}+\mathrm{i}\,h_{l'j}\big), & i{=}k,\; j\neq l,\\[2pt]
			2\!\left[h_{ii'jk'}+h_{ii'j'k}+\mathrm{i}\big(h_{ii'jk}-h_{ii'j'k'}\big)\right], & i{=}l,\\[2pt]
			2\!\left[h_{jj'il'}+h_{jj'i'l}+\mathrm{i}\big(h_{jj'i'l'}-h_{jj'il}\big)\right], & j{=}k,\\[2pt]
			4\big(h_{ik}+\mathrm{i}\,h_{i'k}\big), & j{=}l,\; i\neq k,\\[2pt]
			\upsilon_{1}(i,j,k,l)+\upsilon_{2}(i,j,k,l)
			+\mathrm{i}\big[\upsilon_{3}(i,j,k,l)-\upsilon_{4}(i,j,k,l)\big], & i\neq j\neq k\neq l,
		\end{cases}
		\label{eq:Ezijklst}
	\end{equation}
	که در آن (برای \(1\le i<j\le m,\;1\le k<l\le m\))
	\begin{subequations}
		\begin{align}
			\upsilon_{1}(i,j,k,l) &= h_{ijkl}+h_{ijk'l'}+h_{i'j'kl}+h_{i'j'k'l'}, \\
			\upsilon_{2}(i,j,k,l) &= h_{i'jk'l}-h_{i'jkl'}-h_{ij'k'l}+h_{ij'kl'}, \\
			\upsilon_{3}(i,j,k,l) &= h_{i'jkl}+h_{i'jk'l'}-h_{ij'kl}-h_{ij'k'l'}, \\
			\upsilon_{4}(i,j,k,l) &= h_{ijk'l}-h_{ijkl'}+h_{i'j'k'l}-h_{i'j'kl'}.
		\end{align}
	\end{subequations}

	
	
	\paragraph{میانگین.}
	با یادآوری
	\[
	T_{R}'=\frac{1}{nm(m-1)}\,T_{R}
	=\frac{1}{2m(m-1)}\sum_{i<j}\big|\hat r_{ij}\big|^{2},
	\]
	میانگین تحت \(H_{1}\) برابر است با
	\begin{equation}
		\mu_{1}
		=\frac{1}{2m(m-1)}\sum_{i<j}\mathbb{E}\!\big[|\hat r_{ij}|^{2}\big]
		=\frac{1}{2m(m-1)}\sum_{i<j} g_{ij},
	\end{equation}
	که
	\begin{equation}
		g_{ij}
		=\frac{4}{n^{2}}\Big[n+A_{n,2}\big(h_{ij}^{2}+h_{i'j}^{2}\big)\Big].
	\end{equation}
	\textit{\color{BrickRed}{تذکر (تصحیح):}}
	در متن مقاله برای \(A_{n,m}\) (تعداد جایگشت‌های بدون تکرارِ انتخاب \(m\) عنصر از \(n\) عنصر) به‌اشتباه فرمول دیگری نوشته شده بود. تعریف درست که در این‌جا به کار می‌بریم به صورت زیر است‌:
	$$
	A_{n,m}\;=\;\frac{n!}{(n-m)!}
	$$
	که در این اثبات 
	$A_{n,2}=n(n-1)$
	
	\paragraph{واریانس.}
	واریانس \(T_{R}'\) برابر است با
	\begin{equation}
		\sigma_{1}^{2}
		=\frac{1}{4m^{2}(m-1)^{2}}
		\sum_{i<j}\sum_{k<l}\!\left(
		\mathbb{E}\!\big[|\hat r_{ij}|^{2}\,|\hat r_{kl}|^{2}\big]
		-\mathbb{E}\!\big[|\hat r_{ij}|^{2}\big]\mathbb{E}\!\big[|\hat r_{kl}|^{2}\big]\right).
	\end{equation}
	چون مشاهدات زمانی مستقل‌اند، هرگاه
	\(\delta_{t_{1}t_{2}}+\delta_{t_{3}t_{4}}\ge 1\) یا \(t_{1}\neq t_{2}\neq t_{3}\neq t_{4}\)،
	داریم
	\begin{equation}
		\mathbb{E}\!\big[z_{ij}(t_{1})z_{ij}^{*}(t_{2})z_{kl}(t_{3})z_{kl}^{*}(t_{4})\big]
		=\mathbb{E}\!\big[z_{ij}(t_{1})z_{ij}^{*}(t_{2})\big]\,
		\mathbb{E}\!\big[z_{kl}(t_{3})z_{kl}^{*}(t_{4})\big].
	\end{equation}
	در نتیجه
	\begin{equation}
		\begin{aligned}
			\mathbb{E}\!\big[|\hat r_{ij}|^{2}\,|\hat r_{kl}|^{2}\big]
			&=\mathbb{E}\!\big[|\hat r_{ij}|^{2}\big]\,
			\mathbb{E}\!\big[|\hat r_{kl}|^{2}\big]
			+\frac{1}{n^{4}}\!\!\sum_{(t_{1},t_{2},t_{3},t_{4})\in\mathcal{T}}
			\!\!\mathbb{E}\!\big[z_{ij}(t_{1})z_{ij}^{*}(t_{2})z_{kl}(t_{3})z_{kl}^{*}(t_{4})\big]\\
			&\hspace{2.6cm}
			-\frac{1}{n^{4}}\!\!\sum_{(t_{1},t_{2},t_{3},t_{4})\in\mathcal{T}}
			\!\!\mathbb{E}\!\big[z_{ij}(t_{1})z_{ij}^{*}(t_{2})\big]\,
			\mathbb{E}\!\big[z_{kl}(t_{3})z_{kl}^{*}(t_{4})\big],
		\end{aligned}
	\end{equation}
	که در آن
	\[
	\mathcal{T}=\big\{(a,b,c,d)\,\big|\,\delta_{ab}+\delta_{cd}=0,\;\;
	\delta_{ac}+\delta_{ad}+\delta_{bc}+\delta_{bd}\ge 1\big\}. \qquad (107)
	\]
	پس از جایگذاری \eqref{eq:Ezij} و \eqref{eq:Ezijkl} و \eqref{eq:Ezijklst} و ساده‌سازی، خواهیم داشت
	\begin{equation}
		\mathbb{E}\!\big[|\hat r_{ij}|^{2}\,|\hat r_{kl}|^{2}\big]
		=\mathbb{E}\!\big[|\hat r_{ij}|^{2}\big]\,
		\mathbb{E}\!\big[|\hat r_{kl}|^{2}\big]
		+ f_{ijkl}-g_{ijkl},
	\end{equation}
	که
	\begin{equation}
		g_{ijkl}=\frac{32(n-1)(2n-3)}{n^{3}}\,
		\big(h_{ij}^{2}+h_{i'j}^{2}\big)\big(h_{kl}^{2}+h_{k'l}^{2}\big),
	\end{equation}
	و
	\begin{equation}
		f_{ijkl}=
		\begin{cases}
			\tau_{1}(i,j), & i{=}k,\; j{=}l,\\[2pt]
			\tau_{2}(i,j,l), & i{=}k,\; j\neq l,\\[2pt]
			\tau_{2}(i,j,k), & i{=}l,\\[2pt]
			\tau_{2}(j,i,l), & j{=}k,\\[2pt]
			\tau_{2}(j,i,k), & j{=}l,\; i\neq k,\\[2pt]
			\tau_{3}(j,i,k,l), & i\neq j\neq k\neq l,
		\end{cases}
	\end{equation}
	با
	\begin{equation}
		\tau_{1}(i,j)=\frac{16}{n^{4}}A_{n,2}\!\left[1+h^{2}_{ii'jj'}\right]
		+\frac{32}{n^{4}}A_{n,3}\!\left[\big(h_{ij}^{2}+h_{i'j}^{2}\big)+h_{ii'jj'}\big(h_{ij}^{2}-h_{i'j}^{2}\big)\right],
	\end{equation}
	\begin{equation}
		\begin{aligned}
			\tau_{2}(i,j,k)=\frac{4}{n^{4}}A_{n,2}&\Big(4\big(h_{jk}^{2}+h_{jk'}^{2}\big)
			+\big(h_{ii'jk'}+h_{ii'j'k}\big)^{2}
			+\big(h_{ii'jk}-h_{ii'j'k'}\big)^{2}\Big)\\
			+\frac{16}{n^{4}}A_{n,3}&\Big(
			\big(h_{ii'jk}-h_{ii'j'k'}\big)(h_{ik}h_{i'j}+h_{ij}h_{i'k})
			+\big(h_{ii'jk'}+h_{ii'j'k}\big)(h_{ik}h_{ij}-h_{i'j}h_{i'k})\\
			&\quad+2h_{jk}(h_{ik}h_{ij}+h_{i'j}h_{i'k})
			+2h_{jk'}(h_{ik}h_{i'j}-h_{ij}h_{i'k})\Big),
		\end{aligned}
	\end{equation}
	و
	\begin{equation}
		\begin{aligned}
			\tau_{3}(i,j,k,l)=\frac{2}{n^{4}}A_{n,2}\sum_{t=1}^{4}\upsilon_{t}^{2}(i,j,k,l)
			+\frac{16}{n^{4}}A_{n,3}\Big(
			&\upsilon_{1}(i,j,k,l)\,h_{ij}h_{kl}
			+\upsilon_{2}(i,j,k,l)\,h_{i'j}h_{k'l}\\
			&+\upsilon_{3}(i,j,k,l)\,h_{kl}h_{i'j}
			+\upsilon_{4}(i,j,k,l)\,h_{ij}h_{k'l}
			\Big).
		\end{aligned}
	\end{equation}
	
	در نهایت،
	\begin{equation}
		\sigma_{1}^{2}
		=\frac{1}{4m^{2}(m-1)^{2}}
		\sum_{i<j}\sum_{k<l}\big(f_{ijkl}-g_{ijkl}\big),
	\end{equation}
	که همان بیان واریانس \(\,T_{R}'\,\) تحت \(H_{1}\) است. 
	
\end{اثبات}

مشابه حالت $H_{0}$، تابع توزیع تجمعی (CDF) آماره‌ی $T_{R}'$ تحت $H_{1}$ نیز می‌تواند با یک توزیع بتا تقریب زده شود:
\begin{equation}
	\Pr\{T_{R}' < \gamma\} \;\approx\;
	\frac{\Gamma(\alpha_{1}+\beta_{1})}{\Gamma(\alpha_{1})\,\Gamma(\beta_{1})}\,
	B(\gamma;\alpha_{1},\beta_{1}),
\end{equation}
که در آن
\begin{align}
	\alpha_{1}&=\frac{\mu_{1}\big(\mu_{1}-\mu_{1}^{2}-\sigma_{1}^{2}\big)}{\sigma_{1}^{2}},
\end{align}
\begin{align}
	\beta_{1}&=\frac{(1-\mu_{1})\big(\mu_{1}-\mu_{1}^{2}-\sigma_{1}^{2}\big)}{\sigma_{1}^{2}}.
\end{align}
