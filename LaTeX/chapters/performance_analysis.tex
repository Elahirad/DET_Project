\فصل{تحلیل عملکرد}

در این بخش، افت کارایی در آشکارسازی هنگام استفاده از مبدل‌های آنالوگ–دیجیتال یک‌بیتی \لر{1-bit ADCs}
 در مقایسه با مبدل‌های $\infty$-بیتی بررسی می‌شود.  
توجه کنید که آشکارساز $\infty$-بیتی EMR در دسته‌ی آزمون‌های \لر{sphercity} قرار می‌گیرد؛ آزمون‌هایی که هم استقلال بین متغیرهای تصادفی و هم برابری واریانس‌های آن‌ها را در نظر می‌گیرند.  
با این حال، به دلیل از دست رفتن اطلاعات دامنه در حالت یک‌بیتی، مقایسه‌ی واریانس‌ها غیرممکن می‌شود.  
بنابراین، نتیجه‌ی ما با آزمون \لر{LMPIT}\پانویس{Locally Most Powerful Invariant Test} مقایسه می‌شود.  
در حقیقت، وقتی SNR پایین باشد، درایه‌های قطری ماتریس کوواریانس تقریباً برابر می‌شوند و آزمون \لر{sphercity} کارایی نزدیک به آزمون استقلال دارد، که این موضوع با شبیه‌سازی در \مرجع{huang2015performance} نشان داده شده است.

\قسمت{حالت \لر{$\infty$-bit}}

مسئله‌ی آشکارسازی برای $\infty$-بیت ADC مطابق زیر است.  
\begin{align}
	\mathcal{H}_0 &: \mathbf{R}_\mathbf{x} = \mathrm{diag}(\sigma_{w_1}, ..., \sigma_{w_m}), \notag \\
	\mathcal{H}_1 &: \mathbf{R}_\mathbf{x} \neq \mathrm{diag}(\sigma_{w_1}, ..., \sigma_{w_m}) \notag
\end{align}
\لر{LMPIT} برای این مسئله به صورت زیر بیان می‌شود \مرجع{ramirez2013locally}:
\begin{equation}
	T_{L}=n\,\mathrm{tr}\!\left[\big(\hat{\mathbf{R}}_{\mathbf{x}}\,\mathrm{Diag}(\hat{\mathbf{R}}_{\mathbf{x}})^{-1}-\mathbf{I}_{m}\big)^{2}\right]
	\;\underset{H_{0}}{\overset{H_{1}}{\gtrless}}\;\gamma_{L}, \label{eq:LMPIT}
\end{equation}
که توزیع مجانبی آن در \مرجع{xiao2018approximate} تحلیل شده است:
\begin{equation}
	T_{L}\sim
	\begin{cases}
		\chi^{2}_{k}, & H_{0},\\[6pt]
		\chi^{2}_{k}(\delta_{\infty}^{2}), & H_{1},
	\end{cases}
\end{equation}
که در آن $k=m^{2}-m$ و
$
	\delta_{\infty}^{2}=n\,\mathrm{tr}\!\big[(\mathbf{P}_{\mathbf{x}}-\mathbf{I}_{m})^{2}\big]=2n\|\theta\|^{2}
$
\قسمت{حالت تک‌بیت}

در بخش قبل، توزیع آماره‌ی $T_{R}$ را با توزیع بتا تقریب زدیم.  
با این حال، این تقریب برای مقایسه با آشکارسازهای $\infty$-بیتی جهت تحلیل افت کارایی مناسب نیست.  
بنابراین، در این‌جا یک تقریب جدید برای توزیع $T_{R}$ در SNR پایین استخراج می‌کنیم که در قالب توزیع کای-دو غیرمرکزی بیان می‌شود.  

ابتدا می‌نویسیم:
\begin{equation}
	T_{R}=\|\tilde{\mathbf{r}}_{\mathrm{sc}}\|^{2},
\end{equation}
که در آن 
\(\tilde{\mathbf{r}}_{\mathrm{sc}}=\sqrt{\tfrac{n}{2}}\,\tilde{\mathbf{r}}\)،  
و 
\(\tilde{\mathbf{r}}=\big[\mathrm{Re}(\hat{\mathbf{r}})^{T},\;\mathrm{Im}(\hat{\mathbf{r}})^{T}\big]^{T}\).  
خود
$\hat{\mathbf{r}}$
نیز قبلا به صورت زیر تعریف شد.
\[
\hat{\mathbf{r}}=\big[\hat r_{1,2},\hat r_{1,3},\hat r_{2,3},\ldots,\hat r_{m-1,m}\big]^{T}
\]
\begin{قضیه}
در SNR پایین که $\theta$ از مرتبه‌ی $\mathcal{O}(n^{-1/2})$ است،  
بردار تصادفی
$\tilde{\mathbf{r}}_{\mathrm{sc}}$
به صورت مجانبی از یک توزیع گاوسی چندبعدی حقیقی با میانگین و کوواریانس زیر پیروی می‌کند:
\begin{align}
	\mathbb{E}[\tilde{\mathbf{r}}_{\mathrm{sc}}]
	&=\frac{2}{\pi}\sqrt{2n}\,\theta+\mathcal{O}(n^{-1/2}),\\
	\mathbf{R}_{\tilde{\mathbf{r}}_{\mathrm{sc}}}
	&=\mathbf{I}_{m^{2}-m}+\mathcal{O}(n^{-1/2}).
\end{align}
\end{قضیه}
\begin{اثبات}
	\qquad \\
	\textbf{میانگین و واریانس $\tilde{\mathbf{r}}_{\mathrm{sc}}$} \\
	برای اختصار، هر بردار با اندیس \لر{\texttt{sc}} به‌صورت مقیاس‌یافته تعریف می‌شود:
	\begin{equation}
		\mathbf{r}_{\mathrm{sc}} \triangleq \sqrt{\frac{n}{2}}\,\mathbf{r}.
	\end{equation}
	بنابراین
	\[
	\tilde{\mathbf{r}}_{\mathrm{sc}}
	=
	\begin{bmatrix}
		\mathrm{Re}\!\big(\hat{\mathbf{r}}_{\mathrm{sc}}\big)^{\!T} &
		\mathrm{Im}\!\big(\hat{\mathbf{r}}_{\mathrm{sc}}\big)^{\!T}
	\end{bmatrix}^{\!T},
	\quad
	\hat{\mathbf{r}}_{\mathrm{sc}}
	=
	\big[\,(\hat r_{1,2})_{\mathrm{sc}}, (\hat r_{1,3})_{\mathrm{sc}}, (\hat r_{2,3})_{\mathrm{sc}}, \ldots, (\hat r_{m-1,m})_{\mathrm{sc}} \,\big]^{T}.
	\]
	
	با فرض \(\theta=\mathcal{O}(n^{-1/2})\)، بسط تیلور \(p(\tilde{\mathbf{y}}(t);\theta)\) پیرامون \(\theta_{0}=\mathbf{0}\) به صورت زیر است.
	\begin{align}
		p(\tilde{\mathbf{y}}(t);\theta)
		&= p(\tilde{\mathbf{y}}(t);\theta_{0})
		+ \theta^{T}\!\left.\frac{\partial p(\tilde{\mathbf{y}}(t))}{\partial \theta}\right|_{\theta=\theta_{0}}
		+ \mathcal{O}(n^{-1}) \notag \\
		&= \frac{1}{2^{2m}}
		+ \frac{1}{2^{2m-1}\pi}\sum_{1\le i<j\le m}\!\!\mathrm{Re}\!\big(z_{ij}(t)\big)\,\rho_{ij}
		+ \frac{1}{2^{2m-1}\pi}\sum_{1\le i<j\le m}\!\!\mathrm{Im}\!\big(z_{ij}(t)\big)\,\rho_{i'j}
		+ \mathcal{O}(n^{-1}),
	\end{align}
	که در آن با استفاده از \(\rho_{i'j'}=\rho_{ij}\)، \(\rho_{ij'}=-\rho_{i'j}\) و \(\mathrm{diag}(\mathrm{Im}(\mathbf{P}_{x}))=\mathbf{0}\) را مشابه \eqref{eq:74} و \eqref{eq:75} می‌توان نوشت
	\begin{equation}
		p(\tilde{\mathbf{y}}(t);\theta)
		= \frac{1}{2^{2m}}
		+ \frac{1}{2^{2m-1}\pi}\sum_{1\le i<j\le 2m}\tilde y_i(t)\tilde y_j(t)\,\rho_{ij}
		+ \mathcal{O}(n^{-1}).
	\end{equation}
	سپس می‌توان نتیجه گرفت که :
	\begin{equation}
		p\big(\tilde y_a(t),\tilde y_b(t);\theta\big)
		= \frac{1}{4}+\frac{1}{2\pi}\tilde y_a(t)\tilde y_b(t)\,\rho_{ab}+\mathcal{O}(n^{-1}),
	\end{equation}
	\begin{align}
		&p\big(\tilde y_a(t),\tilde y_b(t),\tilde y_c(t),\tilde y_d(t);\theta\big)
		= \frac{1}{16}
		+ \notag \\
		&\frac{1}{8\pi}\!\left[
		\tilde y_a\tilde y_b\,\rho_{ab}+\tilde y_a\tilde y_c\,\rho_{ac}+\tilde y_a\tilde y_d\,\rho_{ad}
		+\tilde y_b\tilde y_c\,\rho_{bc}+\tilde y_b\tilde y_d\,\rho_{bd}+\tilde y_c\tilde y_d\,\rho_{cd}
		\right] +\mathcal{O}(n^{-1})
	\end{align}
	که 
	$1\leq a \neq b \neq c \neq d \leq 2m$ \\
	در نتیجه
	\begin{align}
		\mathbb{E}\!\big[\tilde y_a(t)\tilde y_b(t)\big]
		&= \sum_{\tilde{y}_e(t)=\pm1 \atop e=a,b} \tilde{y}_a(t)\tilde{y}_b(t)p\big(\tilde y_a(t),\tilde y_b(t);\theta\big)  = \frac{2}{\pi}\rho_{ab}+\mathcal{O}(n^{-1}),\\
		\mathbb{E}\!\big[\tilde y_a(t)\tilde y_b(t)\tilde y_c(t)\tilde y_d(t)\big]
		&=\sum_{\tilde{y}_e(t)=\pm1 \atop e=a,b,c,d} \tilde{y}_a(t)\tilde{y}_b(t)\tilde{y}_c(t)\tilde{y}_d(t)p\big(\tilde y_a(t),\tilde y_b(t),\tilde{y}_c(t),\tilde{y}_d(t);\theta\big) \notag \\
		&= \mathcal{O}(n^{-1}),
	\end{align}
	وقتی تمام اندیس‌های
	${a,b,c,d}$
	یکسان باشند، می‌توان با استفاده از \eqref{eq:B1}، می‌توان 
	$\mathbb{E}\!\big[\tilde y_a(t)\tilde y_b(t)\tilde y_c(t)\tilde y_d(t)\big]$
	را ساده کرد. پس می‌توان نشان داد که :
	\begin{align}
		\mathbb{E}\!\left[\mathrm{Re}\big((\hat r_{ij})_{\mathrm{sc}}\big)\right]
		&= \sqrt{\tfrac{n}{2}}\,
		\mathbb{E}\!\left[\mathrm{Re}(\hat r_{ij})\right] \notag \\[6pt]
		&= \sqrt{\tfrac{n}{2}}\,
		\mathbb{E}\!\left[\frac{1}{n}\sum_{t=1}^{n}\big(\tilde y_{i}(t)\tilde y_{j}(t)
		+ \tilde y_{i'}(t)\tilde y_{j'}(t)\big)\right] \notag \\[6pt]
		&= \sqrt{\tfrac{n}{2}}\,
		\left[\frac{2}{\pi}\rho_{ij} + \frac{2}{\pi}\rho_{i'j'} + \mathcal{O}(n^{-1})\right] \notag \\[6pt]
		&= \frac{2\sqrt{2n}}{\pi}\,\rho_{ij} + \mathcal{O}(n^{-1/2}),
	\end{align}
	
	\begin{align}
		\mathbb{E}\!\left[\mathrm{Im}\big((\hat r_{ij})_{\mathrm{sc}}\big)\right]
		&= \sqrt{\tfrac{n}{2}}\,
		\mathbb{E}\!\left[\mathrm{Im}(\hat r_{ij})\right] \notag \\[6pt]
		&= \sqrt{\tfrac{n}{2}}\,
		\mathbb{E}\!\left[\frac{1}{n}\sum_{t=1}^{n}\big(\tilde y_{i'}(t)\tilde y_{j}(t)
		- \tilde y_{i}(t)\tilde y_{j'}(t)\big)\right] \notag \\[6pt]
		&= \sqrt{\tfrac{n}{2}}\,
		\left[\frac{2}{\pi}\rho_{i'j} - \frac{2}{\pi}\rho_{ij'} + \mathcal{O}(n^{-1})\right] \notag \\[6pt]
		&= \frac{2\sqrt{2n}}{\pi}\,\rho_{i'j} + \mathcal{O}(n^{-1/2}).
	\end{align}
	پس میانگین بردار
	\begin{equation}
		\mathbb{E}\!\left[\tilde{\mathbf{r}}_{\mathrm{sc}}\right]
		= \frac{2\sqrt{2n}}{\pi}\,\theta+\mathcal{O}(n^{-1/2}).
	\end{equation}
	
	از استقلال زمانی، برای \(t_1\neq t_2\) داریم
	\begin{equation}
		\mathbb{E}\!\big[\tilde y_a(t_1)\tilde y_b(t_1)\tilde y_c(t_2)\tilde y_d(t_2)\big]
		= \mathbb{E}\!\big[\tilde y_a(t_1)\tilde y_b(t_1)\big]\,
		\mathbb{E}\!\big[\tilde y_c(t_2)\tilde y_d(t_2)\big].
	\end{equation}
	اکنون کوواریانس‌های مؤلفه‌ها را محاسبه می‌کنیم. با تعریف \(z_{ij}(t)=y_i(t)y_j^{*}(t)\) داریم :
	\begin{align}
		&\mathbb{E}\!\left[\mathrm{Re}\big((\hat r_{ij})_{\mathrm{sc}}\big)\,
		\mathrm{Re}\big((\hat r_{kl})_{\mathrm{sc}}\big)\right]
		- \mathbb{E}\!\left[\mathrm{Re}\big((\hat r_{ij})_{\mathrm{sc}}\big)\right]\,
		\mathbb{E}\!\left[\mathrm{Re}\big((\hat r_{kl})_{\mathrm{sc}}\big)\right] \notag \\[6pt]
		&= \tfrac{n}{2}\,\mathbb{E}\!\left[\mathrm{Re}(\hat r_{ij})\,\mathrm{Re}(\hat r_{kl})\right]
		- \tfrac{n}{2}\,\mathbb{E}\!\left[\mathrm{Re}(\hat r_{ij})\right]\,
		\mathbb{E}\!\left[\mathrm{Re}(\hat r_{kl})\right] \notag \\[6pt] 
		&= \tfrac{1}{2n}\,
		\mathbb{E}\!\left[
		\sum_{t_{1}=1}^{n}\sum_{t_{2}=1}^{n}
		\mathrm{Re}\!\big(z_{ij}(t_{1})\big)\,
		\mathrm{Re}\!\big(z_{kl}(t_{2})\big)
		\right] - \tfrac{1}{2n}\,
		\mathbb{E}\!\left[\sum_{t_{1}=1}^{n}\mathrm{Re}\!\big(z_{ij}(t_{1})\big)\right]\,
		\mathbb{E}\!\left[\sum_{t_{2}=1}^{n}\mathrm{Re}\!\big(z_{kl}(t_{2})\big)\right] \notag \\[6pt]
		&= \tfrac{1}{2n}\sum_{t=1}^{n}\mathbb{E}\!\left[
		\mathrm{Re}\!\big(z_{ij}(t)\big)\,
		\mathrm{Re}\!\big(z_{kl}(t)\big)\right]
		- \tfrac{1}{2n}\sum_{t=1}^{n}\mathbb{E}\!\left[\mathrm{Re}\!\big(z_{ij}(t)\big)\right]\,
		\mathbb{E}\!\left[\mathrm{Re}\!\big(z_{kl}(t)\big)\right] \notag \\[6pt]
		&=\begin{cases}
			1-\tfrac{8}{\pi^{2}}\rho_{ij}^{2}+\mathcal{O}(n^{-1}), & i{=}k,~j{=}l,\\[2pt]
			\tfrac{2}{\pi}\rho_{jl}-\tfrac{8}{\pi^{2}}\rho_{ij}\rho_{il}+\mathcal{O}(n^{-1}), & i{=}k,~j\neq l,\\[2pt]
			\tfrac{2}{\pi}\rho_{jk}-\tfrac{8}{\pi^{2}}\rho_{ij}\rho_{ki}+\mathcal{O}(n^{-1}), & i{=}l,\\[2pt]
			\tfrac{2}{\pi}\rho_{il}-\tfrac{8}{\pi^{2}}\rho_{ij}\rho_{jl}+\mathcal{O}(n^{-1}), & j{=}k,\\[2pt]
			\tfrac{2}{\pi}\rho_{ik}-\tfrac{8}{\pi^{2}}\rho_{ij}\rho_{kj}+\mathcal{O}(n^{-1}), & j{=}l,~i\neq k,\\[2pt]
			-\tfrac{8}{\pi^{2}}\rho_{ij}\rho_{kl}+\mathcal{O}(n^{-1}), & i\neq j\neq k\neq l,
		\end{cases}
	\end{align}
	که با توجه به \(\theta=\mathcal{O}(n^{-1/2})\) معادله بالا به صورت زیر ساده می‌شود:
	\begin{equation}
		\mathbb{E}\!\left[\mathrm{Re}\big((\hat r_{ij})_{\mathrm{sc}}\big)\,
		\mathrm{Re}\big((\hat r_{kl})_{\mathrm{sc}}\big)\right]
		- \mathbb{E}\!\left[\mathrm{Re}\big((\hat r_{ij})_{\mathrm{sc}}\big)\right]\,
		\mathbb{E}\!\left[\mathrm{Re}\big((\hat r_{kl})_{\mathrm{sc}}\big)\right]
		=
		\begin{cases}
			1+\mathcal{O}(n^{-1}), & i=k,\; j=l, \\[6pt]
			\mathcal{O}(n^{-1/2}), & \text{otherwise}.
		\end{cases}
	\end{equation}
	که 
	$1 \leq i < j \leq m$ 
	و
	$1 \leq k < l \leq m$ \\
	به طرز مشابه خواهیم داشت :
	\begin{equation}
		\mathbb{E}\!\left[\mathrm{Re}\big((\hat r_{ij})_{\mathrm{sc}}\big)\,
		\mathrm{Im}\big((\hat r_{kl})_{\mathrm{sc}}\big)\right]
		- \mathbb{E}\!\left[\mathrm{Re}\big((\hat r_{ij})_{\mathrm{sc}}\big)\right]\,
		\mathbb{E}\!\left[\mathrm{Im}\big((\hat r_{kl})_{\mathrm{sc}}\big)\right]
		= \mathcal{O}(n^{-\tfrac{1}{2}}).
	\end{equation}
	و
	\begin{equation}
		\mathbb{E}\!\left[\mathrm{Im}\big((\hat r_{ij})_{\mathrm{sc}}\big)\,
		\mathrm{Im}\big((\hat r_{kl})_{\mathrm{sc}}\big)\right]
		- \mathbb{E}\!\left[\mathrm{Im}\big((\hat r_{ij})_{\mathrm{sc}}\big)\right]\,
		\mathbb{E}\!\left[\mathrm{Im}\big((\hat r_{kl})_{\mathrm{sc}}\big)\right]
		=
		\begin{cases}
			1+\mathcal{O}(n^{-1}), & i=k,\; j=l, \\[6pt]
			\mathcal{O}(n^{-1/2}), & \text{otherwise}.
		\end{cases}
	\end{equation}
	
	
	
	
	در نتیجه ماتریس کوواریانس \(\tilde{\mathbf{r}}_{\mathrm{sc}}\) در SNR پایین
	\begin{equation}
		\mathbf{R}_{\tilde{\mathbf{r}}_{\mathrm{sc}}}=\mathbf{I}_{m^{2}-m}+\mathcal{O}(n^{-1/2}). 
	\end{equation}
	
	\textbf{اثبات گوسی بودن $\tilde{\mathbf{r}}_{\mathrm{sc}}$}
	
	برای اتمام اثبات قضیهٔ ۴ نشان می‌دهیم \(\tilde{\mathbf{r}}_{\mathrm{sc}}\) به‌صورت مجانبی، گاوسی است. از نسخهٔ چندمتغیرهٔ قضیۀ حد مرکزی استفاده می‌کنیم: \\
	\textbf{لم} (\emph{Multivariate CLT}).
	اگر \(\mathbf{s}=\sum_{t=1}^{n}\mathbf{b}_{t}\)، که \(\mathbf{b}_{t}\in\mathbb{R}^{d\times 1}\) مستقل و با میانگین صفرند، آنگاه برای \(n\to\infty\)، \(\mathbf{s}\) با میانگین صفر و کوواریانس \(\mathbf{C}\) مجانبی گاوسی است هرگاه
	\begin{equation}
		\lim_{n\to\infty}\sum_{t=1}^{n}\mathbb{E}\!\left[\left\lVert \mathbf{C}^{-1/2}\mathbf{b}_{t}\right\rVert^{3}\right]=0. 
	\end{equation}
	
	برای به‌کارگیری لم، تعریف می‌کنیم
	\begin{align}
		\tilde{\mathbf{z}}_{t}&=\sqrt{\frac{1}{2n}}
		\begin{bmatrix}
			\mathrm{Re}(\mathbf{z}_{t})^{T} & \mathrm{Im}(\mathbf{z}_{t})^{T}
		\end{bmatrix}^{T}, \\
		\mathbf{z}_{t}&=\big[z_{1,2}(t), z_{1,3}(t),\ldots,z_{m-1,m}(t)\big]^{T},\\
		\mathbf{b}_{t}&=\tilde{\mathbf{z}}_{t}-\mathbb{E}[\tilde{\mathbf{z}}_{t}].
	\end{align}
	با استفاده از \eqref{eq:Ezij} داریم
	\begin{equation}
		\mathbb{E}[\tilde{\mathbf{z}}_{t}]=\Big(\frac{8}{n\pi^{2}}\Big)^{\!1/2}\!\arcsin(\theta),
		\quad\text{(اعمالِ \(\arcsin\) به‌صورت درایه‌به‌درایه)}.
	\end{equation}
	همچنین
	\begin{align}
		\mathbf{s}&=\sum_{t=1}^{n}\mathbf{b}_{t}, \\
		\tilde{\mathbf{r}}_{\mathrm{sc}}&=\mathbf{s}+\mathbb{E}[\tilde{\mathbf{r}}_{\mathrm{sc}}], \\
		\mathbb{E}[\tilde{\mathbf{r}}_{\mathrm{sc}}]&=\sum_{t=1}^{n}\mathbb{E}[\tilde{\mathbf{z}}_{t}]
		=\Big(\frac{8n}{\pi^{2}}\Big)^{\!1/2}\!\arcsin(\theta),
	\end{align}
	و \(\mathbf{C}=\mathbf{R}_{\tilde{\mathbf{r}}_{\mathrm{sc}}}\).
	
	یادآوری می‌کنیم که نامساوی کوشی-شوارتز به صورت زیر است :
	\begin{equation}
		\left\lVert \mathbf{c}_i\mathbf{b}_t\right\rVert^2 \leq \left\lVert\mathbf{c}_i\right\rVert^2 \left\lVert\mathbf{b}_t\right\rVert^2
	\end{equation}
	حال با استفاده از این نامساوی داریم :
	\begin{align}
		\left\lVert \mathbf{C}^{-1/2}\mathbf{b}_{t}\right\rVert^{3}
		&=\left(\sum_{i}\lVert \mathbf{c}_{i}\mathbf{b}_{t}\rVert^{2}\right)^{\!3/2}
		\le \left(\sum_{i}\lVert \mathbf{c}_{i}\rVert^{2}\lVert \mathbf{b}_{t}\rVert^{2}\right)^{\!3/2}
		=\left\lVert \mathbf{C}^{-1/2}\right\rVert^{3}\,\lVert \mathbf{b}_{t}\rVert^{3}. 
	\end{align}
	از نامساوی 
	$\mathbb{E}[f(x)] \geq \mathbb{E}[g(x)]$
	برای 
	$f(x) \geq g(x) \geq 0$
	استفاده می‌کنیم. داریم :
	\begin{equation}
		\mathbb{E}[\left\lVert \mathbf{C}^{-1/2}\mathbf{b}_{t}\right\rVert^{3}] \leq \left\lVert \mathbf{C}^{-1/2}\right\rVert^{3} \mathbb{E}[\lVert \mathbf{b}_{t}\rVert^{3}]
	\end{equation}
	با استفاده از این نامساوی و این نکته که 
	$\left\lVert \mathbf{C}^{-1/2}\mathbf{b}_{t}\right\rVert^{3}$
	محدود است، کافی است نشان دهیم که
	\begin{equation}
		\lim_{n\to\infty}\sum_{t=1}^{n}\mathbb{E}\!\left[\lVert \mathbf{b}_{t}\rVert^{3}\right]=0,
	\end{equation}
	تا شرط لم برقرار باشد. با قانون متوازی‌الاضلاع داریم
	\begin{align}
		\lVert \mathbf{b}_{t}\rVert^{2}+\lVert \tilde{\mathbf{z}}_{t}+\mathbb{E}[\tilde{\mathbf{z}}_{t}]\rVert^{2}
		&=2\big(\lVert \tilde{\mathbf{z}}_{t}\rVert^{2}+\lVert \mathbb{E}[\tilde{\mathbf{z}}_{t}]\rVert^{2}\big)
		\;\Rightarrow\;
		\lVert \mathbf{b}_{t}\rVert^{2}\le 2\big(\lVert \tilde{\mathbf{z}}_{t}\rVert^{2}+\lVert \mathbb{E}[\tilde{\mathbf{z}}_{t}]\rVert^{2}\big).
	\end{align}
	سپس داریم :
	\begin{align}
		\max \lVert \mathbf{b}_{t} \rVert^{2}
		&\;\le\;
		\max 2\Big( \lVert \tilde{\mathbf{z}}_{t} \rVert^{2}
		+ \lVert \mathbb{E}[\tilde{\mathbf{z}}_{t}] \rVert^{2}\Big)
		= \max 2\left(\frac{m(m-1)}{n} + \frac{8}{n\pi^{2}}\lVert \arcsin \theta \rVert^{2}\right) \notag \\[6pt]
		&= \frac{2m(m-1)}{n} + \frac{16}{n\pi^{2}}\max_{\theta}\lVert \arcsin \theta \rVert^{2}
		= \frac{6m(m-1)}{n}.
	\end{align}
	
	در نتیجه
	\begin{align}
		\lim_{n \to \infty} \sum_{t=1}^{n} 
		\mathbb{E}\!\left[ \lVert \mathbf{b}_{t} \rVert^{3} \right]
		&\;\le\;
		\lim_{n \to \infty} \sum_{t=1}^{n} \max \lVert \mathbf{b}_{t} \rVert^{3} \notag \\[6pt]
		&= \lim_{n \to \infty} \sum_{t=1}^{n} 
		\left[ \big(\max \lVert \mathbf{b}_{t} \rVert^{2}\big)^{3/2} \right]
		\;\le\; [6m(m-1)]^{3/2}\,\lim_{n \to \infty} n^{-1/2}
		= 0.
	\end{align}
	
	و بنابراین طبق لم، \(\tilde{\mathbf{r}}_{\mathrm{sc}}\) مجانبی گاوسی است.\\
	اکنون، اثبات قضیه به اتمام رسید.
	
\end{اثبات}
با توجه به نتایج بالا نتیجه می‌گیریم:
\begin{equation}
	T_{R}\sim
	\begin{cases}
		\chi^{2}_{k}, & H_{0},\\[6pt]
		\chi^{2}_{k}(\delta_{1}^{2}), & H_{1},
	\end{cases} \label{eq:chi_approx}
\end{equation}
که در آن
\begin{equation}
	\delta_{1}^{2}=\frac{8n}{\pi^{2}}\|\theta\|^{2}
	=\frac{4}{\pi^{2}}\delta_{\infty}^{2}.
\end{equation}

بنابراین می‌توان نتیجه گرفت که افت کارایی در SNR پایین تقریباً برابر است با:
\[
10\log_{10}\!\left(\sqrt{\frac{\delta_{\infty}^{2}}{\delta_{1}^{2}}}\right)\;\approx\;2~\mathrm{dB}.
\]
به بیان دیگر، این افت کارایی را می‌توان با افزایش اندازه‌ی نمونه‌ها به میزان
$
\frac{\delta_{\infty}^{2}}{\delta_{1}^{2}}
=\frac{\pi^{2}}{4}\approx 2.47
$
جبران کرد.  

شایان ذکر است که در \مرجع{xiao2022onebit} افت \لر{2 dB} تنها با افزایش تعداد نمونه‌ها به اندازه‌ی \linebreak
$\pi/2\approx 1.57$ 
برابر جبران‌پذیر گزارش شده بود.  
در مقایسه با نتیجه‌ی این مقاله، روشن می‌شود که بازده \لر{non-coherent accumulation} برابر با ریشه‌ی دوم بازده \لر{coherent accumulation} است.
