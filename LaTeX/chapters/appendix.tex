
\فصل{کدهای نمایش نهایی نتایج}
کدهای این قسمت از توابع طراحی شده استفاده کرده و نتایج نهایی را نمایش می‌دهند و در پوشه‌ی اصلی \لر{MATLAB} قرار دارند.
توابع مورد استفاده در این قسمت، در پیوست‌های بعدی قرار گرفته است.
\قسمت{ارزیابی ویژگی \لر{CFAR}\label{app:eval-cfar}
(
\texttt{eval\_cfar.m}
)
}
\lr{\inputminted[frame=single, breaklines]{matlab}{code/eval_cfar.m}}

\قسمت{ارزیابی  \لر{False Alarm}\label{app:false_alarm}
	(
	\texttt{eval\_false\_alarm.m}
	)
}
\lr{\inputminted[frame=single, breaklines]{matlab}{code/eval_false_alarm.m}}
\pagebreak
\قسمت{ارزیابی آشکارسازی\label{app:detection}
	(
	\texttt{eval\_detection.m}
	)
}
\lr{\inputminted[frame=single, breaklines]{matlab}{code/eval_detection.m}}

\قسمت{ارزیابی \لر{ROC}\label{app:roc}
		(
	\texttt{eval\_roc.m}
	)
	}
	\lr{\inputminted[frame=single, breaklines]{matlab}{code/eval_roc.m}}
	
	
	\قسمت{ارزیابی آشکارسازی بر حسب \لر{SNR}\label{app:detection-vs-snr}
		(
		\texttt{eval\_detection\_SNR.m}
		)
	}
	\lr{\inputminted[frame=single, breaklines]{matlab}{code/eval_detection_SNR.m}}


\فصل{توابع اصلی}
توابع این قسمت برای  محاسبه احتمالات، نمایش نتایج و تحلیل آن‌ها استفاده می‌شود و در پوشعه‌ی \لر{functions} موجود در پوشه‌ی \لر{MATLAB} قرار دارند.
توابع استفاده شده در این قسمت در پیوست بعدی قرار گرفته است.

\قسمت{توابع تحلیل نتایج}
توابع این قسمت برای جمع‌آوری محاسبات و تحلیل نتایج استفاده می‌شوند.

\زیرقسمت{تابع تحلیل \لر{detection}
	(\texttt{detection\_analysis})
}
این تابع، اولا با شبیه‌سازی مونت‌کارلو احتمال آشکارسازی را به دست می‌آورد و ثانیا با استفاده از توزیع‌های به دست آمده احتمالات رامحاسبه می‌کند و در کنار نمودار شبیه سازی  قرار می‌دهد. 
\lr{\inputminted[frame=single, breaklines]{matlab}{code/functions/detection_analysis.m}}

\زیرقسمت{تابع تحلیل \لر{False Alarm}
	(\texttt{false\_alarm\_analysis})
}
این تابع نیز، همانند تابع قبلی است؛ با این تفاوت که برای احتمال \لر{false alarm} است.
\lr{\inputminted[frame=single, breaklines]{matlab}{code/functions/false_alarm_analysis.m}}


\زیرقسمت{تابع شبیه‌سازی احتمال آشکارسازی بر حسب \لر{SNR}
	(
	\texttt{compute\_detection\_prob\_vs\_SNR}
	)
}
این تابع، برای یک احتمال \لر{false alarm} ثابت، احتمال آشکارسازی را برای \لر{SNR}های مختلف به دست می‌آورد.

\lr{\inputminted[frame=single, breaklines]{matlab}{code/functions/compute_detection_prob_vs_SNR.m}}

\pagebreak
\قسمت{توابع شبیه‌سازی احتمالات}
توابع این قسمت، با اانجام شبیه‌سازی مونت‌کارلو و استفاده از آمارگان معرفی شده، احتمال آشکارسازی و \لر{false alarm} را محاسبه می‌کنند.
\زیرقسمت{شبیه‌سازی احتمال \لر{False Alarm} آشکارساز \لر{Rao} تک‌بیتی
	(
	\texttt{compute\_one\_bit\_rao\_false\_alarm\_prob.m}
	)
}
\lr{\inputminted[frame=single, breaklines]{matlab}{code/functions/compute_one_bit_rao_false_alarm_prob.m}}

\زیرقسمت{شبیه‌سازی احتمال \لر{False Alarm} آشکارساز \لر{EMR} تک‌بیتی
	(
	\texttt{compute\_one\_bit\_emr\_false\_alarm\_prob.m}
	)
}
\lr{\inputminted[frame=single, breaklines]{matlab}{code/functions/compute_one_bit_emr_false_alarm_prob.m}}
\pagebreak
\زیرقسمت{شبیه‌سازی احتمال آشکارسازی آشکارساز \لر{Rao} تک‌بیتی
	(
	\texttt{compute\_one\_bit\_rao\_detection\_prob.m}
	)
}
\lr{\inputminted[frame=single, breaklines]{matlab}{code/functions/compute_one_bit_rao_detection_prob.m}}

\قسمت{توابع شبیه‌سازی \لر{ROC}}

توابع این قسمت نیز با انجام شبیه‌سازی مونت‌کارلو و استفاده از آمارگان معرفی شده، \لر{ROC} را محاسبه می‌کنند.

\زیرقسمت{شبیه‌سازی \لر{ROC} آشکارساز \لر{Rao} تک‌بیتی
	(
	\texttt{compute\_one\_bit\_rao\_roc.m}
	)
}
\lr{\inputminted[frame=single, breaklines]{matlab}{code/functions/compute_one_bit_rao_roc.m}}


\زیرقسمت{شبیه‌سازی \لر{ROC} آشکارساز \لر{EMR} تک‌بیتی
	(
	\texttt{compute\_one\_bit\_emr\_roc.m}
	)
}
\lr{\inputminted[frame=single, breaklines]{matlab}{code/functions/compute_one_bit_emr_roc.m}}



\فصل{توابع کمکی}
این توابع در پوشه‌ی \لر{utilities} موجود در پوشه‌ی \لر{MATLAB} قرار گرفته‌اند و برای محاسبات استفاده می‌شوند.
\قسمت{تابع محاسبه‌ی P (\texttt{compute\_P})}
از تابع زیر برای محاسبه \لر{P}  بر اساس رابطه‌ی \ref{eq:P} استفاده می‌شود.

\lr{\inputminted[frame=single, breaklines]{matlab}{code/utilities/compute_P.m}}
\pagebreak
\قسمت{توابع دریافت $R_s$ و $R_w$
	(\texttt{get\_Rs}
	و
	\texttt{get\_Rw}
	)
}
توابع زیر برای ساخت و دریافت متغیرهای $R_s$ و $R_w$ استفاده می‌شوند.
\lr{\inputminted[frame=single, breaklines]{matlab}{code/utilities/get_Rs.m}}
\lr{\inputminted[frame=single, breaklines]{matlab}{code/utilities/get_Rw.m}}
\قسمت{تابع دریافت $H$
	(\texttt{get\_H})
}
\lr{\inputminted[frame=single, breaklines]{matlab}{code/utilities/get_H.m}}
\pagebreak
\قسمت{توابع دریافت پارامترهای توزیع‌های بتا و کای-دو
	(
	\texttt{get\_beta\_parameters}
	و
	\texttt{get\_chi\_parameters}
	)
}
از توابع زیر برای دریافت پارامترهای توابع بتا و کای استفاده می‌شود. لازم به ذکر است که برای محاسبات از روابط \ref{eq:beta_approx}، \ref{eq:beta_approx2} و \ref{eq:chi_approx} استفاده شده است.
\lr{\inputminted[frame=single, breaklines]{matlab}{code/utilities/get_beta_parameters.m}}
\lr{\inputminted[frame=single, breaklines]{matlab}{code/utilities/get_chi_parameters.m}}
\قسمت{تابع نرمالیزاسیون آستانه‌ها
	(\texttt{get\_normalized\_thresholds})
}
از تابع زیر برای نرمالیزاسیون آستانه ها بر اساس رابطه‌ی \ref{eq:thr_norm} استفاده شده است.
\lr{\inputminted[frame=single, breaklines]{matlab}{code/utilities/get_normalized_thresholds.m}}
\pagebreak
\قسمت{تابع محاسبه‌ی آماره‌ی \لر{Rao}
	(\texttt{compute\_rao\_statistic})
}
از این تابع برای محاسبه‌ی آماره‌ی \لر{‌Rao} بر اساس رابطه‌ی \ref{eq:rao} استفاده می‌شود.
\lr{\inputminted[frame=single, breaklines]{matlab}{code/utilities/compute_rao_statistic.m}}

\قسمت{تابع محاسبه‌ی آماره‌ی \لر{EMR} تک‌بیتی
	(\texttt{compute\_emr\_statistic})
}
از این تابع برای محاسبه‌ی آماره‌ی \لر{‌EMR} تک‌بیتی بر اساس رابطه‌ی \ref{eq:EMR_onebit} استفاده می‌شود.
\lr{\inputminted[frame=single, breaklines]{matlab}{code/utilities/compute_emr_statistic.m}}

\قسمت{تابع محاسبه‌ی آماره‌ی \لر{EMR} $\infty$
	(\texttt{compute\_inf\_emr\_statistic})
}
از این تابع برای محاسبه‌ی آماره‌ی \لر{‌EMR} در حالت \لر{$\infty$-bit} بر اساس رابطه‌ی \ref{eq:EMR_inf} استفاده می‌شود.
\lr{\inputminted[frame=single, breaklines]{matlab}{code/utilities/compute_inf_emr_statistic.m}}

\قسمت{تابع محاسبه‌ی آماره‌ی \لر{LMPIT}
	(\texttt{compute\_LMPIT\_statistic})
}
از این تابع برای محاسبه‌ی آماره‌ی \لر{‌LMPIT} بر اساس رابطه‌ی \ref{eq:LMPIT} استفاده می‌شود.
\lr{\inputminted[frame=single, breaklines]{matlab}{code/utilities/compute_LMPIT_statistic.m}}
