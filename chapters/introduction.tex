
\فصل{مقدمه}

\لر{Spectrum Sensing} یک نیاز اساسی برای تخصیص منابع طیفی در شبکه‌های \ \لر{Cognitive Radio(CR)} به صورت پویا است؛ 
به این صورت که که مسئول پیدا کردن کانال‌های خالی \ (به عنوان \ \لر{Spectrum Holes} هم شناخته می‌شوند)
است.
در این پروژه، ما روشی از \ \لر{Spectrum Sensing}
را که برای مبدل‌های آنالوگ به دیجیتال (\لر{ADC}) تک‌بیتی طراحی شده است را بررسی خواهیم کرد.

\قسمت{چرا تک‌بیتی؟}

در خیلی از سناریوها، وظیفه‌ی \ \لر{Spectrum Sensing} مانیتور کردن کانال‌های باندوسیع است؛
که این، به معنای نیاز به نمونه‌برداری سریع است. از طرفی نیز روش‌های معمول \ \لر{Spectrum Sensing}،
نیاز به کوانتیزاسیون با دقت بالا برای رسیدن به عملکرد ایده‌آل دارند.
ترکیب سرعت بالای نمونه‌برداری و دقت بالای کوانتیزاسیون باعث مصرف انرژی بالایی می‌شود
و از لحاظ عملی مشکل ایجاد خواهد کرد. \\
یک روش موثر برای حل این مشکل، کم کردن دقت کوانتیزاسیون است؛ به معنای دقیق‌تر
استفاده از تنها یک بیت برای مبدل آنالوگ به دیجیتال است.
\لر{ADC}های تک‌بیتی تنها از یک مقایسه‌گر برای انجام نمونه‌برداری و کوانتیزاسیون استفاده می‌کنند؛ که
مزیت‌هایی مانند نرخ نمونه‌برداری بالا، پیچیدگی سخت‌افزار کمتر و مصرف انرژی کمتر را به ارمغان می‌آورند.
به عنوان مثال برای نرخ نمونه‌برداری \ \لر{3.2 GSPS/s}، یک \ \لر{ADC} 8 بیتی 
\lr{105$m$Watt} 
توان مصرف می‌کند. در حالی که این عدد برای \ \لر{ADC} تک بیتی
\lr{20$\mu$Watt}
است.
افت عملکرد ناشی از کاهش دقت کوانتیزاسیون، تنها در حدود 
\lr{2dB ($\pi/2$)}
در \ \لر{SNR}های پایین است که با افزایش نمونه‌ها با ضریب 
\lr{$\pi/2$} قابل جبران است.
توضیحات بالا، میل به استفاده از روش‌های تک‌بیتی را توجیه می‌کند.


\قسمت{آشکارسازی کور}
بسیاری از روش‌های آشکارسازی تک‌بیتی فرض را بر در دسترس بودن اطلاعات پیشین از جمله توان نویز،
اطلاعات کانال و ویژگی‌های سیگنال می‌گذارند.
اما این مقاله بر روی \ \لر{Spectrum Sensing} تک‌بیتی در عدم حضور اطلاعات پیشین
یا اصطلاحا آشکارسازی کور کار می‌کند که با نام \ \لر{Blind Spectrum Sensing} شناخته می‌شود. در این حالت، \ \لر{PMF}\پانویس{Probability mass function} مشاهدات تک‌بیتی، برابر حاصل‌ضرب احتمالات \ \لر{Orthant} می‌شود که فرم بسته ندارد؛ پس نیاز به روش‌های عددی مثل \ \لر{GLRT}\پانویس{Generalized likelihood ratio test} برای طراحی آشکارساز وجود دارد. از طرفی روش‌های عددی، هزینه‌ی محاسباتی و زمانی بالایی دارند که در تضاد با \لر{Spectrum Sensing} ساده است که ما به آن علاقه‌مندیم. در نتیجه، خواسته‌ی ما، یک آشکارساز با معادلات فرم بسته است.

\قسمت{آشکارساز EMR}
آشکارسازی تحت عنوان \لر{One-Bit EMR}\پانویس{One-Bit eigenvalue moment ratio} در \مرجع{zhao2021onebit} با الهام از آشکارساز \لر{EMR‌} \مرجع{huang2015eigenvalue} 
(\lr{$\infty$-bit})
 معرفی شد که نسبت به حالت 
\lr{$\infty$}
در حدود \لر{3 dB} ضعیف‌تر بود، اما بعد از آن اثبات شد که در \لر{SNR}های پایین، افت عملکرد، تنها در حدود \لر{2 dB} است.

علت زیاد بودن افت عملکرد به این دلیل است که برای \لر{one-bit EMR}، ابتدا قسمت‌های حقیقی و موهومی مشاهدات در کنار هم قرار داده می‌شود و سپس \لر{EMR} مربوط به ماتریس کوواریانس (دارای مقادیر حقیقی) محاسبه می‌شود. این امر باعث نادیده گرفته شدن خاصیت \لر{Circularity} سیگنال‌های کوانتیزه‌شده می‌شود.
در این مقاله، آشکارسازی برای مشاهدات تک‌بیتی معرفی شده است که به وسیله تست \لر{Rao} به دست می‌آيد و جهت افزایش کارایی، خاصیت \لر{Circularity} را نیز در نظر می‌گیرد. نتیجه، به صورت \لر{EMR} مرتبه دوم ماتریس \لر{sample covariance} (دارای مقادیر مختلط) است (برعکس روش قبلی که ماتریس کوواریانس گسترش یافته حقیقی را استفاده می‌کرد).

\قسمت{توزیع آشکارساز}
برای تایید افت \لر{2 dB} و مقایسه با رقبای
\lr{$\infty$-bit}
نیاز به یافتن توزیع آشکارساز است. در این مقاله، توزیع‌هایی تقریبی فقط برای حالت حضور نویز و \لر{SNR} پایین معرفی شده است که قابل مقایسه با توزیع های معرفی شده در \مرجع{xiao2018approximate} هستند. جزئیات این مورد در بخش‌های بعد توضیح داده خواهد شد.






